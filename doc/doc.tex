\documentclass[12pt]{article}

\usepackage{ocamlweb}
\usepackage{fullpage}
\usepackage[latin1]{inputenc}
\usepackage[T1]{fontenc}
\usepackage{latexsym}
\usepackage{amssymb}
\usepackage{url}
\usepackage{epsf}
\usepackage{verbatim}
\usepackage[chapter]{/homes/owre/tex/tocbibind}
\usepackage[bookmarks=true,hyperindex=true,colorlinks=true,backref=true,pagebackref=true]{hyperref}

\newcommand{\replace}[2]{#1[#2]}
\newcommand{\norm}[2]{\mathit{norm}(#1)(#2)}
\newcommand{\lookup}[2]{\mathit{lookup}(#1)(#2)}
\newcommand{\can}[2]{\mathit{can}(#1)(#2)}
\newcommand{\dom}[1]{\mathit{dom}(#1)}
\newcommand{\compose}[2]{#1; #2}
\newcommand{\refsec}[1]{\textbf{\ref{#1}}}
\newcommand{\fullrefsec}[1]{Section~\textbf{\ref{#1}} page~\pageref{#1}}
\newcommand{\arrlk}[2]{#1[#2]}
\newcommand{\arrup}[3]{#1[#2:=#3]}
\newcommand{\comm}{\textit{command}}
\newcommand{\term}{\textit{term}}
\newcommand{\prop}{\textit{proposition}}

\title{ICS Manual (Version 0.001)}

\author{The ICS group\\[0.3em]
  {\small Computer Science Laboratory, SRI International} \\[-0.2em]
  {\small 333 Ravenswood Avenue, Menlo Park, CA 94025, USA} \\[-0.2em]
  {\small\texttt{ruess@csl.sri.com}}}

\date{}

\begin{document}

\maketitle
\tableofcontents

\section{Introduction}\label{sec:intro}

ICS (Integrated Canonizer and Solver) is a
decision procedure developed at SRI International.
It efficiently decides formulas in a useful combination
of theories, and it provides an API that makes
it suitable for use in applications with highly dynamic
environments such as proof search or symbolic simulation.

The theory decided by ICS is a quantifier-free,
first-order theory of equality with terms built from
  \begin{itemize}
  \item uninterpreted function symbols,
  \item arithmetic operators including inequalities and number predicates,
  \item propositional connectives,
  \item array lookup and update,
  \item sets operators, and
  \item operators on fixed-sized bitvectors.
  \end{itemize}
This theory is particularly interesting for many applications in the realm
of software and hardware verification.  Combinations of a multitude
of datatypes occur naturally in system specifications and the use of
uninterpreted function symbols has proven to be essential
for many real-world verifications.

The core of ICS is a congruence closure procedure~\cite{RuessShankar2001}
for the theory of equality and disequality with both uninterpreted and
interpreted function symbols.  The concepts of canonization and solving
have been extended to include inequalities over linear arithmetic
terms and propositional logic~\cite{FORS2001}\@.
The theory currently includes:
  \begin{itemize}
  \item The usual propositional constants {\tt true}, {\tt false}
        and connectives
        {\tt not}, {\tt \&}, {\tt |}, {\tt =>}, {\tt <=>}\@.
  \item Equality ({\tt =}) and disequality ({\tt /=})\@.
  \item Rational constants and the arithmetic
        operators {\tt +}, {\tt *}, {\tt -}\@;
        multiplication is restricted to
        multiplication by constants.
        Arithmetic predicates include an integer test and
        the usual inequalities {\tt <}, {\tt <=}, {\tt >}, {\tt >=}\@.
  \item Lookup {\tt $a$[$x$]} and update {\tt $a$[$x$:=$t$]}
        operations for a functional array $a$\@.
  \item The constant sets ({\tt empty}, {\tt full}), 
        set membership ({\tt $x$ in $s$}),
        and set operators, including
        complement ({\tt compl($s$)}),
        union ({\tt $s_1$ union $s_2$}),
        and intersection ({\tt $s_1$ inter $s_2$})\@.
  \item Fixed-sized bitvectors including constants such as {\tt 0b101},
        concatenation ({\tt conc[3,4]($b_1$,$b_2$)}),
        extraction ({\tt $b$[$i$:$j$]}), bit-wise operations
        like bit-wise conjunction  ({\tt bv\_and[n]($b_1$,$b_2$)}),
        and built-in arithmetic relations such as {\tt add[n]($b_1$,$b_2$,$b$)}\@.
        This latter constraint encodes the fact that the sum of the unsigned
        interpretations of $b_1$ and $b_2$ equals the unsigned
        interpretation of $b$.  Fixed-sized bitvectors are
        decided using the techniques described in~\cite{Moeller98:FMCAD}\@.
  \end{itemize}
ICS is capable of deciding formulas such as
  \begin{itemize}
  \item {\tt x+2 = y => f(a[x:=3][y-2]) = f(y-x+1)}
  \item {\tt f(y-1)-1 = y+1 \&  f(x)+1 = x-1 \& x+1 = y => false}
  \item {\tt f(f(x)-f(y)) /= f(z) \&  y <= x \& y >= x+z \& z >= 0 => false}
  \end{itemize}
These formulas contain uninterpreted function symbols
such as {\tt f} and interpreted symbols drawn from the theories of
arithmetic and the functional arrays.  The list of
interpreted theories above is open-ended in the sense that new
theories can be added to ICS as long as they are canonizable and
algebraically solvable. The modular design of ICS---both the
underlying algorithms and their implementation---supports such
extensions.

One of the main problems in employing decision procedures
effectively is due to the fact that verification conditions
usually depend on large contexts.  In addition, these
contexts change frequently in applications such as
symbolic simulation or backtracking proof search.
Consequently, decision procedure systems that
are effective in these domains must not only be able to
build up contexts incrementally but they must
also support efficiently switching between a multitude
of contexts.  ICS meets these criteria in that all of its
main algorithm work incrementally and the data
structures for representing contexts are persistent,
that is, operations on data structures do not alter the
previous values of data and {\em undo} operations
are therefore basically for free.

ICS is implemented in \textsf{Ocaml}~\cite{ObjectiveCaml},
which offers satisfactory run-time performance,
efficient garbage collection, and interfaces well with other
languages like C\@.  The implementation of ICS is based on optimization
techniques such as hash-consing~\cite{JCF2000}
and efficient data structures like Patricia trees~\cite{OkasakiGill98}
for representing sets and maps efficiently.

There is a well-defined API for manipulating ICS terms, asserting
formulas to the current database, switching between databases,
and functions for canonizing terms.
% and there is a functional and an imperative API\@.
This API is packaged as a C library, an \textsf{Ocaml} module,
and a CommonLisp interface. The C library API, for example, has been
used to connect ICS with PVS~\cite{PVS}\@, and both an interaction and
a batch processing capability have been built using this API\@.

Using ICS as the underlying decision procedure of PVS, we
experience speed-ups of several orders of magnitude (compared
with the PVS default decision procedures) for selected problems.
and for typical problems we are usually able to process several
thousand theorems every second.
The efficiency and scalability of ICS in processing formulas,
the richness of its API, and its ability for fast context-switching
should make it possible to use it as a black box for discharging
verification conditions not only in a theorem proving context
but also in a multitude of applications like static analysis,
abstract interpretation, extended type checking,
symbolic simulation, model checking, or compiler
optimization.

\subsection{Availability}

ICS is available free of charge under the PVS license. It is
also included in the upcoming release of PVS 3.0\@.
The complete sources and documentation of ICS are
available at
  \begin{center}
  \url{www.icansolve.com}
  \end{center}
There one can also find the ICS license.


\subsection{Organization}

This document describes the interfaces and implementation
aspects of the ICS decision procedures. It is organized as
follows.
\begin{center}
  \begin{tabular}{p{10cm}rr}
    Chapter & section & page \\[0.5em]
    \hline\\[0.2em]
    Installation         & \refsec{sec:installation}   & \pageref{sec:installation}\\[0.5em]
    Interactor           & \refsec{sec:interactor}     & \pageref{sec:interactor} \\[0.5em]
    API                  & \refsec{sec:api}            & \pageref{sec:api}    \\[0.5em]
    Terms                & \refsec{sec:terms}          & \pageref{sec:terms}  \\[0.5em]
    Constraints          & \refsec{sec:cnstrnts}       & \pageref{sec:cnstrnts} \\[0.5em]
    Substitution         & \refsec{sec:subst}          & \pageref{sec:subst}\\[0.5em]
    Canonization         & \refsec{sec:can}            & \pageref{sec:can}\\[0.5em]
    Solving              & \refsec{sec:solve}          & \pageref{sec:solve}   \\[0.5em]
    Processing           & \refsec{sec:process}        & \pageref{sec:process}   \\[0.5em]
    Basic Data Structures& \refsec{sec:data.structures}&\pageref{sec:data.structures} \\[0.5mm]
    Interpreted Theories & \refsec{sec:theories}       & \pageref{sec:theories}   \\
    Binary Decision Diagrams   & \refsec{sec:bdd}      & \pageref{sec:bdd}\\[0.5em]
    Tools and Support    & \refsec{sec:misc}               & \pageref{sec:misc}   \\[0.5em]\\
  \end{tabular}
\end{center}

The various modules of the implementation are not presented in
topological order with respect to their inter-dependencies, but
rather in a more logical order. However, it may help to have in mind
the graph of dependencies, which is the following:

%BEGIN LATEX
\input{epsf}
\epsfxsize=15cm
\begin{center}
\epsfbox{./dep.ps}
\end{center}
%END LATEX
%HEVEA\imgsrc{./dep.gif}


\section{Installation}\label{sec:installation}

\paragraph{Distribution.}
The file \texttt{ics.tar.gz} can be downlowded from \url{www.icansolve.com}.
Unpack this file using
  \begin{verbatim}
  > zcat ics.tar.gz | tar xvf -
  \end{verbatim}
This creates a directory \texttt{./ics} with the following files
and directory.\\
  \begin{tabular}{lcl}
   \texttt{Makefile.in}  & : & Template for generating \texttt{Makefile}.  \\
   \texttt{bin/} & : & Binaries \\
   \texttt{configure} & : & Configuration script \\
   \texttt{lib/} & : & Archives and shared object files  \\
   \texttt{tests/} & : & Various benchmarks  \\
   \texttt{README} & : & Short installation guide  \\
   \texttt{doc/} & : & Documentation \\
   \texttt{obj/} & : & Object files \\
   \texttt{src/} & : & Source files \\
   \texttt{ics/} & : & Shell script for invoking ICS interactor\\
  \end{tabular}

The latest version of ICS can be obtained by using the
CVS checkout command
  \begin{verbatim}
  > cvs co ics -d $CVS_root
  \end{verbatim} % $
where \texttt{CVS\_root} is set to
  \begin{verbatim}
  /project/pvs2/cvsroot
  \end{verbatim}

\paragraph{Requirements.}
ICS is written mainly in Ocaml, and it uses uses arbitrary
precision rational numbers from the GNU multi-precision library (GMP)\@.
To compile ICS one needs to install:
  \begin{itemize}
  \item The \textsf{autoconf} package is freely available
        at \url{http://www.gnu.org/software/autoconf/}\@.
  \item \textsf{Ocaml} version 3.00 or later.
        Freely available at \url{http://caml.inria.fr}\@.
  \item \textsf{GNU MP} version 3.0 or later. This package is
        freely available at \url{http://www.swox.com/gmp/}\@.
  \end{itemize}
So far, we have only compiled \textsf{ICS} on \textsf{Linux Redhat 6.0},
but it should be possible to compile it on a wide range of
machines and operating systems.

\paragraph{Installation.}
A configuration \texttt{./configure} is generated from
\texttt{./configure.in} in the ICS home directory using
  \begin{verbatim}
  > autoconf ./configure.in > ./configure
  > chmod +x ./configure
  \end{verbatim}
The configuration script generates a \texttt{Makefile} from the
\texttt{Makefile.in}. 
  \begin{verbatim}
  > ./configure
  \end{verbatim}
Then, {\tt make} compiles ICS on your machine.
  \begin{verbatim}
  > make
  \end{verbatim}
Binaries are placed in \texttt{./bin/\$(ARCH)/} and
the libraries in {\tt ./lib/\$(ARCH)/}, where \texttt{ARCH}
is the architecture guessed by the configuration script.
The build directory is {\tt ./obj/\$(ARCH)/}, and the generated
binary and byte code are put in \texttt{./bin/\$(ARCH)/}.


\section{The ICS interactor}\label{sec:interactor}

The interactor permits to process formulas interactively and to explore
the database.  We give an overview of the capabilities of ICS
using various little examples.

The interactor is started with \texttt{./ics} in the ICS home directory.
\begin{verbatim}
% ics
ICS interpreter. Copyright (c) 2001 SRI International.

> 
\end{verbatim}
The `\texttt{>}' is the prompt and ICS is ready to interpret your commands.
Let us start by checking various tautologies:
\begin{verbatim}
> check ((p => q) => p) => p.
Valid.

> check (x <= y) => (x >= y) => x = y.
Valid.

> check f(f(x)) = x => f(f(f(f(x)))) = x. 
Valid.
\end{verbatim}
One can also use ICS in an incremental way, by adding statements into
its database. The command for that is \texttt{assert}.
\begin{verbatim}
> assert x = 0.

> assert f(x) = 1.
\end{verbatim}
At any point, one can browse the current state with the
\texttt{show} command. This rhs of each equation is
the canonical representative of the equivalence class
containing the rhs.
\begin{verbatim}
> show.
f(0) |-> 1
x |-> 0
\end{verbatim}
At some point, an \texttt{assert} command can fail because an
inconsistency is discovered:
\begin{verbatim}
> assert f(f(x) - 1) = 2.
Inconsistent!
\end{verbatim}
In that case, nothing is added to the state.
One can reset the state to an empty state with the \texttt{reset}
command:
\begin{verbatim}
> reset.          
\end{verbatim}
When you enter complex propositions with \texttt{assert}, they are
stored in the state but their consistency is not always checked
immediately (unless trivial).
\begin{verbatim}
> assert a => (b & c).
Ok.
> assert ~b & a.
Ok.
\end{verbatim}
Therefore, you can end up with an inconsistent state without
notification by ICS.  That is perfectly normal, the \texttt{assert}
command being designed to be fast. But the \texttt{check} command is
complete, and hence you can always check the (in)consistency of the
state with the following command:
\begin{verbatim}
> check false.
Valid.
\end{verbatim}
Similarly, you could have asserted only the first proposition and
checked the validity of the negation of the second one:
\begin{verbatim}
> assert a => (b & c).
Ok.
> check ~(~b & a).
Valid.
\end{verbatim}

\bigskip
The \texttt{ics} program can also be used as an interpreter, in the
following way:
\begin{center}
  \texttt{ics} [\texttt{-s}] [\texttt{-t}] \textit{file}
\end{center}
Option \texttt{-s} prints statistics when the program exits, and
option \texttt{-t} prints timings for each validity check.

Consider, for example, processing
{\tt f(y-1)-1 = y+1}, {\tt f(x)+1 = x-1}, and {\tt x+1 = y}
from left-to-right using the interactive mode of ICS\@.
  \begin{verbatim}
  % ics
  ICS interpreter. Copyright (c) 2001 SRI International.
  Ctrl-d to exit.

  >  assert f(y - 1) - 1 = y + 1.
  \end{verbatim}
This equation is asserted in its solved form as
{\tt y = -2 + f(-1 + y)}\@. This equation is indeed considered
to be in solved form, since {\tt y} on the right-hand side occurs
only in the scope of the uninterpreted {\tt f}\@. Terms
in the database are partitioned
into equivalence classes, and the canonical representative of any
term $t$ with respect to this partition is represented by {\tt can $t$};
for example:
  \begin{verbatim}
  > can -1 + y.
  -3 + f(-1 + y)
  \end{verbatim}
It can be shown that {\tt can $t_1$} is identical to {\tt can $t_2$}
iff the equality $t_1$ = $t_2$ is derivable in the current context.
Now, the second equation is processed
  \begin{verbatim}
  > assert f(x) + 1 = x - 1.
  \end{verbatim}
\vspace{-5mm}
by canonizing it to {\tt 1 + f(x) = -1 + x} and solving this
equation as {\tt x = 2 + f(x)}. Finally, {\tt can x + 1} yields
{\tt 3 + f(x)} and {\tt can y} is {\tt -2 + f(-1 + y)}. Thus,
the third equation is solved as {\tt f(x) = -5 + f(-1 + y)}\@.
Since {\tt f(x) = f(-1 + y)}, using {\tt x = -1 + y} and congruence,
there is a contradiction {\tt -5 = 0}\@.
Indeed, ICS detects this inconsistency, when given the assertion
below.
 \begin{verbatim}
  > assert x + 1 = y.
  Inconsistent!
  \end{verbatim}
\vspace{-5mm}


\subsection{The Command Language}

The ICS command language realizes a {\em ask/tell} interface to
a context consisting of known facts. When invoking
the interactor, this context is empty.

\paragraph{Asserting facts.}
  \begin{center}
  \texttt{assert} \prop \texttt{.}
  \end{center}
A new fact $\prop$ is asserted to the database using the
\texttt{assert} command. There are three possible outcomes:
  \begin{enumerate}
  \item  $\prop$ is inconsistent with respect to the current context.
         In this case, \texttt{assert} leaves the current context
         unchanged and outputs \texttt{Inconsistent!} on the standard
         output.
  \item  $\prop$ is valid in the current context. Again,the
         the current context is left, and now \texttt{Valid!}
         is output.
  \item  Otherwise, in case $\prop$ is satisfiable but not valid,
         the context is modified to include new information obtained
         from $\prop$.
  \end{enumerate}

\paragraph{Checking validity.}
  \begin{center}
  \texttt{check} \prop \texttt{.}
  \end{center}
The validity and inconsistenty of a proposition $\prop$ in the current
context can be established using the \texttt{check} command.  This
command leaves the context unchanged.

\paragraph{Resetting.}
  \begin{center}
  \texttt{reset} \texttt{.}
  \end{center}
Resets the current context to the empty context.

\paragraph{Canonization.}
  \begin{center}
  \texttt{canon} \term \texttt{.}
  \end{center}
Computes the canonical form for its argument $\term$\@ with respect
to the current context.

\paragraph{Displaying the context.}
  \begin{center}
  \texttt{show} \texttt{.}
  \end{center}
Displays the current context on standard output as a list of
equations $\textit{rhs} = \textit{lhs}$\@.  The $\textit{lhs}$
is the representative of the equivalence class containing $\textit{rhs}$\@.

\paragraph{Verbose.}
  \begin{center}
  \texttt{verbose} \textit{int} \texttt{.}
  \end{center}
Determines the amount of trace information displayed on standard output.
Currently, the verbose levels $0$, $1$, and $2$ are supported.



\section{Application Programming Interface}\label{sec:api}

Usually, the capabilites of ICS are not accessed through the interactor
but rather through its application programming interface.  Currently,
we support interfaces for C, Fortran, Lisp, and Ocaml.  We
first describe the Ocaml interface, since the interfaces for the
other programming languages are automatically generated from this one.

\subsection{Caml interface}

 
\newpage
\section*{Library API}

\ocwsection\label{api}
This chapter presents the library public interface, the so-called API
(Application Programming Interface). It is composed of a single module
\textit{Ics}.  


\subsection{C Interface}

The API for the C programming language is generated automatically
from the Ocaml API described above. The C file can be found in
  \begin{verbatim}
  ./obj/$ARCH/ics_stub.c
  \end{verbatim}
% $
This file contains a C function declaration \texttt{ics\_xxx} for each of the interface
function \texttt{xxx} described above.  For example, the definition of
the function \texttt{ics\_mk\_var} for the \texttt{mk\_var} constructor is
given by the following C code.
  \begin{verbatim}
value* ics_mk_var(char* x1) {
  value* ics_mk_var(char* x1) {
  value* r = malloc(sizeof(value));
  register_global_root(r);
  *r = 1;
  *r = callback_exn(*ics_mk_var_rv,copy_string(x1));
  if (!Is_exception_result(*r)) { return r; };
  ocaml_error("ics_mk_var",format_caml_exception(Extract_exception(*r)));
  return (value*)0;
}
 \end{verbatim}
These interface function translate C arguments to Ocaml values, call the
Ocaml function, and translate back the results. In addition, any Ocaml
exceptions are caught and handled by the \texttt{ocaml\_error} function.
Curried signatures of the Ocaml functions are uncurried, and list and
tuple arguments must be build using the constructors of the interface.
The handling of exceptions is determined by the function
\texttt{ocaml\_error}, defined in \texttt{./src/ics\_error.c}; this
function may have to be adjusted for application-specific error handling.

\subsection{Lisp Interface}

The Lisp API for ICS builds on the C interface and uses the foreign
function interface of Allegro Common Lisp 6.0.  For each function
\texttt{xxx} in the API a foreign function declaration \texttt{ics\_xxx}
is generated.  These definitions are collected in the file
 \begin{verbatim}
  ./obj/$ARCH/ics.lisp
  \end{verbatim}
% $
The Lisp interface ensures that the Lisp and the Ocaml garbage collector
cooperate.

\section{Term structure}\label{sec:terms}

\subsection{Term data structure}\label{subsec:terms}

\input{./literal/term.mli}

\input{./literal/term.ml}


  \subsection{Uninterpreted Terms}\label{sec:uninterpreted}

  \subsubsection{Variables}\label{subsec:variables}

  \input{./literal/var.mli}

  \input{./literal/var.ml}

 \subsubsection{Function application and update}\label{subsec:app}

  \input{./literal/app.mli}

  \input{./literal/app.ml}


\section{Constraints}\label{sec:cnstrnts}

\subsection{Interval arithmetic}\label{subsec:interval}

  \input{./literal/interval.mli}

  \input{./literal/interval.ml}

\subsection{Manipulating term constraints}\label{subsec:cnstrnts}

  \input{./literal/cnstrnt.mli}

  \input{./literal/cnstrnt.ml}



\section{Theories}\label{sec:theories}

  This chapter describes the implementation of canonization and solver
  functions for the various theories


 \subsection{Linear Arithmetic}\label{subsec:arithmetic}

    \input{./literal/arith.mli}

    \input{./literal/arith.ml}

  \subsubsection{Rational Polynomials}\label{subsec:bdd}

    \input{./literal/poly.mli}

    \input{./literal/poly.ml}

    \subsubsection{Euclidean Solver}\label{subsec:euclid}

    \input{./literal/euclid.mli}

    \input{./literal/euclid.ml}


 \subsection{Tuples}\label{subsec:tuples}

   \input{./literal/tuple.mli}

   \input{./literal/tuple.ml}


 \subsection{Propositional Logic}\label{subsec:propositional}

 \input{./literal/bool.mli}

 \input{./literal/bool.ml}


 \subsection{Sets}\label{subsec:sets}

 \input{./literal/sets.mli}

 \input{./literal/sets.ml}


 \subsection{Bitvectors}\label{subsec:bv}

  \input{./literal/bv.mli}
  \input{./literal/bv.ml}

  \subsubsection{Concatenation Normal Forms}\label{sec:cnf}

  \input{./literal/conc.mli}

  \input{./literal/conc.ml}


\section{Binary Decision Diagrams}\label{sec:bdd}

\input{./literal/bdd.mli}

\input{./literal/bdd.ml}


\section{Substition and Normalization}\label{sec:subst}

 \input{./literal/subst.mli}

 \input{./literal/subst.ml} 


\section{Simplification and Canonization}\label{sec:can}

 \input{./literal/can.mli}

 \input{./literal/can.ml}

\section{Solving}\label{sec:solve}

 \input{./literal/solve.mli}

 \input{./literal/solve.ml}


\section{Processing}\label{sec:process}

 \subsection{Database}\label{subsec:database}

 \input{./literal/state.mli}

 \input{./literal/state.ml}

 \subsection{Process}\label{subsec:process}

 \input{./literal/process.mli}

 \input{./literal/process.ml}



\section{Basic Data Structures}\label{sec:data.structures}

  \subsection{Hash Consing}\label{subsec:hashcons}

  \input{./literal/hashcons.mli}

  \input{./literal/hashcons.ml}

  \subsection{Set of Terms}\label{subsec:patricia.sets}

  \input{./literal/ptset.mli}

  \input{./literal/ptset.ml}

  \subsection{Term Maps}\label{subsec:patricia.maps}

  \input{./literal/ptmap.mli}

  \input{./literal/ptmap.ml}





\section{Tools and Support}\label{sec:misc}

  \subsection{Tools}\label{subsec:tools}

  This chapter describes the utility modules. The module \textit{Tools}
  contained utility functions used in the implementation. The modules
  \textit{Lexer} and \textit{Parser} provide parsers for terms,
  equations, commands, etc. (we only give the interface for module
  \textit{Lexer}). The module \textit{Test} is a tiny toplevel
  to test the implementation. It allows the user to test the
  canonization and solver functions, and the algorithm.

  \input{./literal/tools.mli}

  \input{./literal/tools.ml}

  \section{Arbitrary Precision Arithmetic}\label{sec:apa}

   \input{./literal/mpa.mli}
 
   \input{./literal/mpa.ml}

  \subsection{Exceptions}\label{subsec:exceptions}

   \input{./literal/exc.mli}

   \input{./literal/exc.ml}

  \subsection{Binary Relations}\label{subsec:binrel}

   \input{./literal/binrel.mli}

   \input{./literal/binrel.ml}

  \subsection{Comparison}\label{subsec:cmp}

   \input{./literal/cmp.mli}

   \input{./literal/cmp.ml}
 
  \subsection{Tracing}\label{subsec:trace}

   \input{./literal/trace.mli}

   \input{./literal/trace.ml}

 \section{Command Interpreter}\label{sec:interpreter}

  \subsection{Commands}\label{subsec:cmd}

  \input{./literal/cmd.mli}
  \input{./literal/cmd.ml}

  \subsection{Read-Eval-Print}\label{subsec:repl}

  \input{./literal/main.ml}

  \subsection{Lexer}\label{subsec:lexer}
 
  \verbatiminput{../src/lexer.mll}

  \subsection{Parser}\label{subsec:parser}
 
  \verbatiminput{../src/parser.mly}

  \subsection{Pretty-Printer}\label{subsec:pretty}

  \input{./literal/pretty.mli}

  \input{./literal/pretty.ml}


  \subsection{Lexical conventions}

  The lexical conventions of ICS are the following:
  \begin{center}
   \begin{tabular}{rrl}
    \textit{ident} & ::= & (\textit{letter}$|$\texttt{\_}) 
                           (\textit{letter}$|$\texttt{\_}$|$\texttt{'}$|$\textit{digit})$\star$
                           \\
    \textit{number} & ::= & (digit)+
  \end{tabular}
  \end{center}
  Comments start with the character \texttt{\%}, up to the end of line.

  \subsection{BNF grammar}

\begin{center}
  \begin{tabular}{rrl}
    \comm & ::= & \texttt{assert} \prop\texttt{;} \dots\texttt{;} 
                  \prop\texttt{.} \\
          & $|$ & \texttt{check} \prop\texttt{.} \\
          & $|$ & \texttt{show}\texttt{.} \\
          & $|$ & \texttt{canon} \term\texttt{.} \\
          & $|$ & \texttt{verbose} \textit{int}\texttt{.} \\
          & $|$ & \texttt{reset.} \\
  \end{tabular}
\end{center}

\begin{center}
  \begin{tabular}{rrl}
    \prop & ::= & \term\ \texttt{=} \term \\
          & $|$ & \term\ \texttt{<>} \term \\
          & $|$ & \term\ \texttt{>} \term \\
          & $|$ & \term\ \texttt{>=} \term \\
          & $|$ & \term\ \texttt{<} \term \\
          & $|$ & \term\ \texttt{<=} \term \\
          & $|$ & \texttt{true} \\
          & $|$ & \texttt{false} \\
          & $|$ & \prop\ \texttt{\&\&} \prop \\
          & $|$ & \prop\ \texttt{||} \prop \\
          & $|$ & \prop\ \texttt{=>} \prop \\
          & $|$ & \texttt{\~{}} \prop \\
          & $|$ & \texttt{if} \prop\ \texttt{then} \prop\ 
                  \texttt{else} \prop\ \texttt{fi} \\
          & $|$ & \ldots
  \end{tabular}
\end{center}

\begin{center}
  \begin{tabular}{rrl}
    \term & ::= & \textit{ident} \\
          & $|$ & \textit{ident}
                  \texttt{(}\term\texttt{,}\dots\texttt{,}\term\texttt{)} \\
          & $|$ & \textit{number} \\
          & $|$ & \term\ \texttt{+} \term \\
          & $|$ & \term\ \texttt{-} \term \\
          & $|$ & \term\ \texttt{*} \term \\
          & $|$ & \term\ \texttt{/} \term \\
          & $|$ & TODO
  \end{tabular}
\end{center}


\vspace{10mm}
{\bf Acknowledgements.}
The algorithms and data structures underlying ICS have been
developed by N. Shankar and Harald Rue{\ss}\@. The core ICS code is
by Jean-Christophe Filli{\^a}tre and Harald Rue{\ss}, and the Lisp
interface has been developed by Sam Owre, Harald Rue{\ss}, and
Jean-Christophe Filli{\^a}tre. The GMP ocaml interface was originally
written by David Monniaux, and adjusted for use with \texttt{ocaml 3.00} by
Jean-Christophe Filli{\^a}tre.

 This document has been automatically produced
from the source code using the literate programming tool \textsf{ocamlweb}%
\footnote{\textsf{Objective Caml} and
  \textsf{ocamlweb} are both freely available, respectively at
  \textsf{http://caml.inria.fr} and
  \textsf{http://www.lri.fr/\~{}filliatr/ocamlweb}}.


\bibliographystyle{alpha}
\addcontentsline{toc}{section}{References}
\bibliography{./biblio}

\end{document}








