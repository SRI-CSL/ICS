\documentclass[12pt]{article}

\usepackage{fullpage}

\begin{document}

\thispagestyle{empty}
\vspace*{4cm}
\begin{center}
   {\Huge\sf ICS} 
   \\[3cm]
   {\large\sf Jean-Christophe Filli\^atre, Sam Owre, Harald Rue\ss, 
   and Natarajan Shankar}
   \\[3cm]
  Computer Science Laboratory, SRI International \\
  333 Ravenswood Avenue, Menlo Park, CA 94025, USA
\end{center}
\vfill\newpage


\section*{Introduction}

ICS is a decision procedure for the combination of 
\begin{itemize}
\item Propositional satisfiability
\item Pure equality over uninterpreted function symbols
\item Equality combined with theories (arithmetic, arrays, tuples,
  sets, bit vectors)
\item Linear arithmetic inequalities
\end{itemize}


\section{The ICS interactive program}

The interactive program is called \texttt{ics}.
It allows you to enter commands and print their results.
Here is a quick introduction to the ICS commands, through various
little examples.

First we start the interactive program:
\begin{verbatim}
mymachine% ics
ICS interpreter. Copyright (c) 2000 SRI International.

> 
\end{verbatim}
The `\texttt{>}' is the prompt and ICS is ready to interpret your commands.
Let us start by checking various tautologies:
\begin{verbatim}
> check ((p => q) => p) => p.
Valid.

> check (x <= y) => (x >= y) => x = y.
Valid.

> check f(f(x)) = x => f(f(f(f(x)))) = x. 
Valid.
\end{verbatim}

You can also use ICS in an incremental way, by adding statements into
its `state'. The command for that is \texttt{assert}.
\begin{verbatim}
> assert x = 0.

> assert f(x) = 1.
\end{verbatim}
At any point, you can have a look at the current state with the
command \texttt{show}.
\begin{verbatim}
> show.
(find)
  f(0) = 1;
  x = 0;
\end{verbatim}
As you can see on this example, ICS is keeping everything in a
canonized form. 
At some point, an \texttt{assert} command can fail because an
inconsistency is discovered:
\begin{verbatim}
> assert f(f(x) - 1) = 2.
Inconsistent!
\end{verbatim}
In that case, nothing is added to the state.
You can reset the state to an empty state with the \texttt{reset}
command:
\begin{verbatim}
> reset.          
\end{verbatim}
When you enter complex propositions with \texttt{assert}, they are
stored in the state but their consistency is not always checked
immediatly (unless trivial).
\begin{verbatim}
> assert a => (b && c).

> assert ~b && a.
\end{verbatim}
Therefore, you can end up with an inconsistent state without
notification by ICS.  That is perfectly normal, the \texttt{assert}
command being designed to be fast. But the \texttt{check} command is
complete, and hence you can always check the (in)consistency of the
state with the following command:
\begin{verbatim}
> check false.
Valid.
\end{verbatim}
Similarly, you could have asserted only the first proposition and
checked the validity of the negation of the second one:
\begin{verbatim}
> assert a => (b && c).

> check ~(~b && a).
Valid.
\end{verbatim}

\bigskip
The \texttt{ics} program can also be used as an interpreter, in the
following way:
\begin{center}
  \texttt{ics} [\texttt{-s}] [\texttt{-t}] \textit{file}
\end{center}
Option \texttt{-s} prints statistics when the program exits, and
option \texttt{-t} prints timings for each validity check.


\section{Installing ICS}

TODO


\section{Using ICS as a library}

TODO


\section{Grammar of the ICS interpreter}

\subsection{Lexical conventions}

The lexical conventions of ICS are the following:
\begin{center}
  \begin{tabular}{rrl}
    \textit{ident} & ::= & (\textit{letter}$|$\texttt{\_}) 
                           (\textit{letter}$|$\texttt{\_}$|$\texttt{'}$|$\textit{digit})$\star$
                           \\
    \textit{number} & ::= & (digit)+
  \end{tabular}
\end{center}
Comments start with the character \texttt{\%}, up to the end of line.

\subsection{BNF grammar}

\newcommand{\comm}{\textit{command}}
\newcommand{\term}{\textit{term}}
\newcommand{\prop}{\textit{proposition}}

\begin{center}
  \begin{tabular}{rrl}
    \comm & ::= & \texttt{assert} \prop\texttt{;} \dots\texttt{;} 
                  \prop\texttt{.} \\
          & $|$ & \texttt{check} \prop\texttt{.} \\
          & $|$ & \texttt{show}\texttt{.} \\
          & $|$ & \texttt{canon} \term\texttt{.} \\
          & $|$ & \texttt{verbose} \textit{int}\texttt{.} \\
          & $|$ & \texttt{reset.} \\
  \end{tabular}
\end{center}

\begin{center}
  \begin{tabular}{rrl}
    \prop & ::= & \term\ \texttt{=} \term \\
          & $|$ & \term\ \texttt{<>} \term \\
          & $|$ & \term\ \texttt{>} \term \\
          & $|$ & \term\ \texttt{>=} \term \\
          & $|$ & \term\ \texttt{<} \term \\
          & $|$ & \term\ \texttt{<=} \term \\
          & $|$ & \texttt{true} \\
          & $|$ & \texttt{false} \\
          & $|$ & \prop\ \texttt{\&\&} \prop \\
          & $|$ & \prop\ \texttt{||} \prop \\
          & $|$ & \prop\ \texttt{=>} \prop \\
          & $|$ & \texttt{\~{}} \prop \\
          & $|$ & \texttt{if} \prop\ \texttt{then} \prop\ 
                  \texttt{else} \prop\ \texttt{fi}
  \end{tabular}
\end{center}

\begin{center}
  \begin{tabular}{rrl}
    \term & ::= & \textit{ident} \\
          & $|$ & \textit{ident}
                  \texttt{(}\term\texttt{,}\dots\texttt{,}\term\texttt{)} \\
          & $|$ & \textit{number} \\
          & $|$ & \term\ \texttt{+} \term \\
          & $|$ & \term\ \texttt{-} \term \\
          & $|$ & \term\ \texttt{*} \term \\
          & $|$ & \term\ \texttt{/} \term \\
          & $|$ & TODO
  \end{tabular}
\end{center}

\end{document}
