
%%% Macros

% trick: since [ and ] are interpreted by ocamlweb, we use the
% following macro
\newcommand{\replace}[2]{#1[#2]}
\newcommand{\norm}[2]{\mathit{norm}(#1)(#2)}
\newcommand{\lookup}[2]{\mathit{lookup}(#1)(#2)}
\newcommand{\can}[2]{\mathit{can}(#1)(#2)}
\newcommand{\dom}[1]{\mathit{dom}(#1)}
\newcommand{\compose}[2]{#1; #2}
\newcommand{\refsec}[1]{\textbf{\ref{#1}}}
\newcommand{\fullrefsec}[1]{Section~\textbf{\ref{#1}} page~\pageref{#1}}
\newcommand{\arrlk}[2]{#1[#2]}
\newcommand{\arrup}[3]{#1[#2:=#3]}

%%% Introduction

\ocwsection
This document presents an implementation of a decision
procedure based on a variation of Shostak's algorithm by Harald Rue\ss\
and Natarajan Shankar.

This implementation is written in \textsf{Objective Caml} and this
document has been automatically produced from the source code using
the literate programming tool \textsf{ocamlweb}%
\footnote{\textsf{Objective Caml} and
  \textsf{ocamlweb} are both freely available, respectively at
  \textsf{http://caml.inria.fr} and
  \textsf{http://www.lri.fr/\~{}filliatr/ocamlweb}}.

This document is organized as follows.
\begin{center}
  \begin{tabular}{p{10cm}rr}
    Chapter & section & page \\[0.5em]
    \hline\\[0.2em]
    Library API         \dotfill & \refsec{api}    & \pageref{api}    \\[0.5em]
    Terms               \dotfill & \refsec{terms}  & \pageref{terms}  \\[0.5em]
    Shostak's algorithm \dotfill & \refsec{algo}   & \pageref{algo}   \\[0.5em]
    Decision procedure  \dotfill & \refsec{dp}     & \pageref{dp}     \\[0.5em]
    Theories            \dotfill & \refsec{theo}   & \pageref{theo}   \\
    ~~ Arithmetic       \dotfill & \refsec{arith}  & \pageref{arith}  \\
    ~~ Tuples           \dotfill & \refsec{tuple}  & \pageref{tuple}  \\
    ~~ Arrays           \dotfill & \refsec{arrays} & \pageref{arrays} \\[0.5em]
    Utilities           \dotfill & \refsec{misc}   & \pageref{misc}   \\[0.5em]
  \end{tabular}
\end{center}    

The various modules of the implementation are not presented in
topological order with respect to their inter-dependencies, but
rather in a more logical order. However, it may help to have in mind
the graph of dependencies, which is the following:

%BEGIN LATEX
\input{epsf}
\epsfxsize=15cm
\begin{center}
\epsfbox{./dep.ps}
\end{center}
%END LATEX
%HEVEA\imgsrc{./dep.gif}
